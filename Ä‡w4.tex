\documentclass[12pt, letterpaper, titlepage]{article}
\usepackage[left=3.5cm, right=2.5cm, top=2.5cm, bottom=2.5cm]{geometry}
\usepackage[MeX]{polski}
\usepackage[utf8]{inputenc}
\usepackage{graphicx}
\usepackage{enumerate}
\usepackage{amsmath} %pakiet matematyczny
\usepackage{amssymb} %pakiet dodatkowych symboli
\title{Ćwiczenie 4}
\author{Paweł Pełszyk}
\date{20 listopada 2022}
\begin{document}
\maketitle

Zadanie 4
\begin{enumerate}[A.]

\item  $$ \sqrt{ \frac{2^{n}}{2_n}} \neq \sqrt[\frac{1}{3}]{1+n} $$

\item $$ \frac{2^{k}}{2^{k+2}}$$

\item $$ \frac{x^{2}} {2^{(x+2)(x-2)^{3}}}$$

\item $$ log_2 \ 2^{8}=8 $$

\item $$ \sqrt[3]{e^{x}-log_2 \ x}$$

\item $$ \lim_{n\to\infty} \sum_{k=1}^{n} \frac{1}{k^{2}}=\frac{\pi^{2}}{6}  $$

\item $$ \int_{2}^{\infty} \frac{1}{log_{2} \ x }\mathrm{d} x = \frac{1}{x}sin x = 1- cos^{2}(x) $$

\item $$ \mathbf{}
\left[ \begin{array}{cccc}

a_{11} & a_{12} & \ldots & a_{1K} \\
a_{21} & a_{22} & \ldots & a_{2K} \\
\vdots & \vdots & \ddots & \vdots \\
a_{K1} & a_{K2} & \ldots & a_{KK} \\
\end{array} \right]
\mathbf{*}
\left[ \begin{array}{c}
x_{1} \\
x_{2} \\
\vdots \\
x_{K} \\
\end{array} \right]
\mathbf{=}
\left[ \begin{array}{c}
b_{1} \\
b_{2} \\
\vdots \\
b_{K} \\
\end{array} \right]
$$

\item $$(a_{1} = a_{1}(x)) \wedge (a_{2}=a_{2}(x)) \wedge \ldots \wedge (a_{k} = a_{k}(x)) \Rightarrow (d=d(u)) $$

\item $$ [x]_{A} = \{y \in U : a(x) = a(y), \forall a\in A\}, \textrm{where the control object} x \in U $$

\item $$ T:[0,1] \textrm{x} [0,1] \rightarrow [0,1] $$

\item $$ \lim_{x\to\infty} \exp(-x) = 0 $$

\item $$ \frac{n!}{k!(n-k)!} = {n \choose k} $$

\item $$ P\left( A=2 \bigg| \frac{A^{2}}{B} > 4 \right) $$

\item $$ S^{C_{i}}(a) = \frac{(\overline{C}_{i}^{a} - \hat{C}_{i}^{a})^{2}}{Z_{\overline{C}_{i}^{a^{2}}} + Z_{\hat{C}_{i}^{a^{2}}}} \textrm{, } a\in A $$

\end{enumerate}

\end{document}