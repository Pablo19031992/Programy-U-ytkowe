\documentclass[12pt, letterpaper, titlepage]{article}
\usepackage[left=3.5cm, right=2.5cm, top=2.5cm, bottom=2.5cm]{geometry}
\usepackage[MeX]{polski}
\usepackage[utf8]{inputenc}
\usepackage{graphicx}
\usepackage{enumerate}
\usepackage{amsmath} %pakiet matematyczny
\usepackage{amssymb} %pakiet dodatkowych symboli
\title{ćwiczenie 2}
\author{Paweł Pełszyk}
\date{15 Października 2023}
\begin{document}
\maketitle

\section{Drożdżówki z rabarbarem i masą sernikową}

\paragraph{Składniki}
\begin{enumerate}[-]
\item  
\underline{200 ml} 
\textbf{mleka}
\item 
\underline{15 g} 
\textbf{drożdży}
\item 
\underline{100g}
 i
\underline{6 łyżek} 
\textbf{cukru}
\item 
\underline {550 g} 
\textbf{mąki pszennej}
\textit{\textbf{typ 500}}
\item 
\underline{3} \textbf{żółtka}
\item 
\underline{2} \textbf{jajka}
\item 
\underline{1/4 łyżeczki} \textbf{soli}
\item 
\underline{200 g}  \textbf{zimnego masła}
\item 
\underline{200 g} \textbf{mąki krupczatki}
\item 
\underline{około 3 pędy} \textbf{rabarbaru}
\item 
\underline{garść} \textbf{dowolnych owoców jagodowych, sezonowych}
\item 
\underline{250 g} \textbf{twarogu}
\item 
\underline{1 opakowanie} \textbf{cukieru waniliowego}
\item 
\underline{2-3 łyżki} \textbf{śmietany}
\item 
\underline{80 g} \textbf{miękkiego masła}
\end{enumerate}
\paragraph {Przygotowanie}
\subparagraph{MASA SEROWA:}
250 g twarogu 2 żółtka, cukier waniliowy 3 łyżki cukru i 2-3 łyżki śmietany wymieszać, zmiksować na gładką masę.
\subparagraph{CIASTO} 
\begin{enumerate}[1]
\item W misce rozpuścić w 200ml mleka 15g drożdży i 4 łyżki cukru.
\item Dodać roztrzepane 2 jajka z 1 żółkiem, wymieszać wszystko na jednolitą masę.
\item Dodać 550 g mąki pszennej i sól, wymieszać aż wszystko się połączy.
\item Przełożyć na oprószoną mąką stolnicę i zagnieść przez ok 3 minuty aż ciasto będzie sprężyste i przestanie kleić się do rąk. W trakcie wyrabiania, szczególnie w początkowej fazie warto delikatnie podsypywać stolnicę i dłonie mąką aby ciasto się nie kleiło.
\item Po tym czasie rozpłaszczyć ciasto, nałożyć 1/3 bardzo miękkiego masła, zagnieść boki ciasta do środka i dalej zagniatać. Gdy masło się wchłonie powtórzyć czynność z kolejną porcją i tak do zużycia całego masła.
\item Ciasto przełożyć z powrotem do miski. Zostawić do wyrośnięcia aż podwoi swoją objętość.
\item W trakcie gdy ciasto wyrasta przygotować kruszonkę. W misce wymieszać 200 g masła, 200g mąki krupczatki i 100g cukru. Rozcierać wszystkie składniki końcówkami palców, aż wszystko się ładnie połączy i powstaną grudki. Wstawić do lodówki. Jeśli nie wykorzystasz całej kruszonki od razu możesz pozostałą część zamrozić.
\item Przygotować owoce. Umyty rabarbar pokroić na 0,5 cm plasterki. Rabarbaru nie obieramy. Dzięki pokrojeniu na małe kawałki skórka nie będzie nam przeszkadzać, a rabarbar zachowa piękną różową barwę.
\item Ciasto po wyrośnięciu podzielić na kawałki o wadze 100g. Każdą uformować w kulkę i odstawić do wyrośnięcia na 15 minut.
\item Bułeczki rozwałkować na 0,5 cm placki i ułożyć na blachach do pieczenia w odstępach. Za pomocą worka cukierniczego wycisnąć na każdą drożdżówkę teochę masy serowej. Możesz też oczywiście zrobi" to za pomocą łyżeczki. Na wierzch położyć owoce i posypać kruszonką. Odstawić do wyrośnięcia na 40 minut.

\end{enumerate}
\end{document}
